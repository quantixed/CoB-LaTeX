\documentclass[vruler,JCS]{COB}
\unsetvruler
\usepackage{boites}
\RequirePackage{siunitx}
% this line causes a l3regex obsolete warning - problem seems Overleaf-side
\RequirePackage[version=4]{mhchem}
%% Units
\DeclareSIUnit\molar{\mole\per\cubic\deci\metre}
\DeclareSIUnit\Molar{\textsc{m}}
\DeclareSIUnit\Units{\textnormal{U}}
%% Hyperlinking
\RequirePackage[hidelinks]{hyperref}
\RequirePackage{url}
\RequirePackage{lipsum}

\begin{document}

\supertitle{Research Article}
%\supertitle{Tools and Techniques}

\title{The title of a really cool paper}

\author{First Author$^{1}$ and Second Author$^{2}$}

\address{\add{1}{First author address}
\add{2}{Second author address}}

\corres{(\email{xxxx@xxxx.xxx.xx})}

\date{Received 1 May 2018; revised 5 December 2018}

\maketitle

\begin{abstract}
Abstract of the paper goes here.
\lipsum[1]
\end{abstract}

\keywords{keyword entry 1, keyword entry 2, keyword entry~3}

\section{Introduction}\label{s:introduction}

\lipsum[2-8]


\section{Results}\label{s:results}

\subsection{Citations and full size figures with legends underneath}

Text is added like this
This is a reference to a published paper \citep{watson_molecular_1953}.
We can cite other things too \citep{tipton_complexities_2019,zheng_genome_2011,alberts_molecular_2002}

This is a new paragraph.
New sentences on a new line.
New sentences on a new line.

% this is how to add a comment
This is a new result.
% this is how to add a figure with the name cells.
As you can see (Figure \ref{fig:cells}).

% full size figure is figure*
\begin{figure*}
\centering
\includegraphics[width=0.75\linewidth]{Figures/temp.png}
\caption{\textbf{These are cells.}\\
(\textbf{A}) This is a regular figure with a legend as a caption underneath. Inset: 3X zoom. Scale bar, \SI{10}{\micro\meter}.}
\label{fig:cells}
\end{figure*}

It is possible to add a one-column Figure like this (Figure \ref{fig:nucleus}).
To add Supplementary Figures you can do either of these things and have them at the end of the end of the paper (Supplementary Figure \ref{suppfig:endosome}).
Or like this (Supplementary Figure \ref{suppfig:lysosome}).

\lipsum[10]

\subsection{Subsections are written like this}

\lipsum[11]

% one-column size figure is figure
\begin{figure}
\centering
\includegraphics[width=0.75\linewidth]{Figures/temp.png}
\caption{\textbf{This is a nucleus.}\\
(\textbf{A}) This is a one-column figure with a legend as a caption underneath.}
\label{fig:nucleus}
\end{figure}

\lipsum[12]

\subsection{Another subsection}

\lipsum[13-14]

\subsection{Another subsection}

\lipsum[13-14]

\subsection{Another subsection}

\lipsum[13-14]

\section{Discussion}\label{s:discussion}

This is the discussion section where you wax lyrical about your findings.
You can put your work in the context of other published work \citep{brenner_uga:_1967}

\lipsum[100-104]

\section{Methods}\label{s:methods}

\subsection{Molecular biology}

Details of plasmids are usually first.
Followed by cell biology section.
We have special units defied for molar and for units, e.g. \SI{1}{\Molar} sucrose, \SI{10}{\Units\per\milli\litre}.
Otherwise use siunitx for everything else. \SI{37}{\degreeCelsius} and what-not.

\subsection{Cell biology}

\lipsum[80]


%%% For acknowledgements
\ack{
Acknowledgements go here.}

%%% For Competing interests
\competing{
The authors declare no competing or financial interests.}

%%% For contribution
\contribution{
Author contributions can be listed here.}

%%% For funding
\funding{
Funding information goes here.}

%%% For data availability
\data{
Insert the Data availability text here.}

%%% For supplementary
\supplementary{
Two supplementary figures are available below.}

% if hand-formatting references use this section
% if using bibTeX use \bibliography
% \section{References}
\newcommand{\newblock}{}
\bibliographystyle{JCS.bst}
\bibliography{example.bib}


\clearpage

\setcounter{figure}{0} % reset figure counter for Supp. Figures
\makeatletter 
\renewcommand{\thefigure}{S\@arabic\c@figure} % make Figure legend start with Figure S
\makeatother


\begin{figure*}
\centering
\includegraphics[width=0.75\linewidth]{Figures/temp.png}
\caption{\textbf{This is an endosome.}\\
(\textbf{A}) This is a supplementary figure shown as a two-column image with a legend underneath.}
\label{suppfig:endosome}
\end{figure*}

\begin{figure}
\centering
\includegraphics[width=0.75\linewidth]{Figures/temp.png}
\caption{\textbf{This is a lysosome.}\\
(\textbf{A}) This is a supplementary figure shown as a one-column image with a legend underneath.}
\label{suppfig:lysosome}
\end{figure}

\end{document}